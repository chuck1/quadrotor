The position control law outputs net jerk in the inertial frame.

j = C_9 chi_5 + C_10 (x_ref - x) + C_11 (x_ref' - v) + C_12 (x_ref'' - a) + x_ref'''

From this jerk, x- and y-components of angular velcity and rate of change of net thrust are calculated.

Ignoring drag, the net force in the global frame is

- f - net force on drone in global frame
- f_g - force from gravity
- f_R - force from rotors in global frame
- f_RB - force from rotors in body frame

f = f_g + f_R

Differentiate in time

f' = f_R' = m j

Rotor force in the inertial and body frame is related by

f_R = q* f_RB q

Differentiate in time

f_R' = q* ( f_RB' + f_RB x \G{\omega} ) q

In our case

\[
\V{f}_{RB} = \begin{bmatrix} 0 \\ 0 \\ f_{RB} \end{bmatrix}
\]

and the above becomes

\[
\V{q} \Vd{f}_R \V{q}^* =
\begin{bmatrix}
-\omega_y f_{zRB} \\
\omega_x f_{zRB} \\
\dot{f}_{zRB}
\end{bmatrix}
\]

The left side is known. First, the derivative of rotor force from the z-component is integrated to get rotor force for the current time step.

\[
f_{zRB}(t_i) = f_{zRB}(t_{i-1}) + \dot{f}_{zRB}
\]

The angular velocity components can then be calculated. These are the reference values for the second controller with outputs torque.

\subsubsection{Time Derivative of quaternion rotation}

In general, if q rotates from y to x frame

\[
\V{y} = \V{q}^* \V{x} \V{q}
\]



\[
\Vd{y} = \Vd{q}^* \V{x} \V{q} + \V{q}^* \Vd{x} \V{q} + \V{q}^* \V{x} \Vd{q}
\]

\[
\Vd{y} = \V{q}^* \G{\omega}^* \V{x} \V{q} + \V{q}^* \Vd{x} \V{q} + \V{q}^* \V{x} \G{\omega} \V{q}
\]

\[
\Vd{y} = \V{q}^* \left( \G{\omega}^* \V{x} + \Vd{x} + \V{x} \G{\omega} \right) \V{q}
\]

\[
\V{q} \Vd{y} \V{q}^* = \G{\omega}^* \V{x} + \Vd{x} + \V{x} \G{\omega}
\]



which leads to the simple result

\[
\G{\omega}^* \V{x} + \Vd{x} + \V{x} \G{\omega} = 
\begin{bmatrix}
0 \\
- \omega_y a_{RB} \\
\omega_x a_{RB} \\
\dot{a}_{RB}
\end{bmatrix}
\]

Notice that $\omega_z$ does not appear because it has no effect on the translation of the vehicle.
In order to solve for $\omega_x$, $\omega_y$, and $a_{RB}$, we need a relationship between $a_{RB}$ and its derivative.
For example, we could use forward differencing.






