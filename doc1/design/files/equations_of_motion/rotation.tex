# Rotation

I omega' = tau - omega x ( I omega )

omega' = I^-1 ( tau - omega x ( I omega ) )

I = [
[I_xx & 0 & 0]
[0 & I_yy & 0]
[0 & 0 & I_zz]
]

for an individual rotor

gamma = omega^2

thrust = k gamma

torque = b gamma

In the body-frame, rotor 0 is on the y+ axis, rotor 2 on the y- axis, rotor 1 on the x- axis, and rotor 3 on the x+ axis. All rotors are equal distance L to the center of mass

tau_B = [
L k ( gamma_0 - gamma_2 ) \\
L k ( gamma_1 - gamma_3 ) \\
b ( gamma_0 - gamma_1 + gamma_2 - gamma_3 )
]

tau_B =
\begin{bmatrix}
Lk & 0  & -Lk & 0 \\
0  & Lk & 0   & -Lk \\
b  & -b & b   & -b
\end{bmatrix}
\begin{bmatrix}
\gamma_1 \\ \gamma_2 \\ \gamma_3 \\ \gamma_4
\end{bmatrix}
\]

\subsubsection{Quaternion}

\[
\dot{\mathbf{q}} = \mathbf{A}_3 \boldsymbol\omega
\]

\[
\ddot{\mathbf{q}} = \dot{\mathbf{A}}_3 \boldsymbol\omega + \mathbf{A}_3 \dot{\boldsymbol\omega}
\]

\[
\ddot{\mathbf{q}}
= \dot{\mathbf{A}}_3 \boldsymbol\omega
+ \mathbf{A}_3 \mathbf{I}^{-1} \left( \boldsymbol\tau - \boldsymbol\omega \times \left( \mathbf{I} \boldsymbol\omega \right) \right)
\]

\subsubsection{Euler}

\[
\boldsymbol\omega
=
\begin{bmatrix}
1 & 0		& -s_{\boldsymbol\theta} \\
0 & c_{\phi}	& c_{\boldsymbol\theta} s_{\phi} \\
0 & -s_{\phi}	& c_{\boldsymbol\theta} s_{\phi}
\end{bmatrix}
\dot{\boldsymbol\theta} 
= \mathbf{A}_5^{-1} \dot{\boldsymbol\theta}
\]

\[
\dot{\boldsymbol\theta} = \mathbf{A}_5 \boldsymbol\omega
\]

\[
\ddot{\boldsymbol\theta} = \dot{\mathbf{A}}_5 \boldsymbol\omega + \mathbf{A}_5 \dot{\boldsymbol\omega}
\]

\[
\ddot{\boldsymbol\theta}
= \dot{\mathbf{A}}_5 \boldsymbol\omega
+ \mathbf{A}_5 \left(
	\mathbf{I}^{-1} \left( \boldsymbol\tau - \boldsymbol\omega \times \left( \mathbf{I} \boldsymbol\omega \right) \right)
\right)
\]

\begin{equation}
\ddot{\boldsymbol\theta} =
\mathbf{S}^{-1} \left(
	\mathbf{I}^{-1} \left( \boldsymbol\tau - \boldsymbol\omega \times \left( I \boldsymbol\omega \right) \right)
	- \dot{\mathbf{S}} \dot{\boldsymbol\theta}
\right)
\label{eq:eom_theta}
\end{equation}










